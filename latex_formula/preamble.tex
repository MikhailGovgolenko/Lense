% !TEX program = lualatex
% !TEX root = main.tex
\documentclass[a4paper,12pt]{article}

% Разметка страницы
\usepackage[a4paper,
  left=2cm, right=2cm,
  top=2cm, bottom=2cm]{geometry}

% ---------------------------
% Языки и шрифты
% ---------------------------
\usepackage{fontspec}
\usepackage[russian,english]{babel}

\setmainfont{CMU Serif}
\setsansfont{CMU Sans Serif}
\setmonofont{CMU Typewriter Text}

% Если нужен Times:
% \setmainfont{Times New Roman}
% \setsansfont{Arial}
% \setmonofont{Courier New}

% Математические шрифты
\usepackage{unicode-math}
\setmathfont{Latin Modern Math}

% Для \mathcal и \mathbb как в обычном LaTeX
\DeclareMathAlphabet{\mathcal}{OMS}{cmsy}{m}{n}
\let\mathbb\relax
\DeclareMathAlphabet{\mathbb}{U}{msb}{m}{n}

% ---------------------------
% Эпиграф
% ---------------------------
\usepackage{epigraph}
\usepackage{ragged2e}
\renewcommand{\textflush}{flushleft}
\renewcommand{\sourceflush}{flushright}

% ---------------------------
% Ссылки и структура
% ---------------------------
\usepackage[
  colorlinks=true,
  linkcolor=blue,
  urlcolor=blue,
  citecolor=blue
]{hyperref}

\usepackage{perpage}
\MakePerPage{footnote}

\usepackage{titlesec}
\titleformat{\section}
  {\normalfont\Large\bfseries\filcenter}
  {\thesection}{1em}{}

\usepackage{enumitem}

% ---------------------------
% Базовые пакеты
% ---------------------------
\usepackage{amsmath}
\usepackage{indentfirst}
\usepackage{float}
\usepackage{array}
\usepackage{multirow}
\usepackage{booktabs}
\usepackage{caption}
\usepackage{manfnt}

% ---------------------------
% Колонтитулы
% ---------------------------
\usepackage{fancyhdr}
\pagestyle{fancy}
\setlength{\headheight}{15pt}
\fancyhead[L]{}
\fancyhead[C]{\nouppercase{\leftmark}}
\fancyhead[R]{}
\renewcommand{\headrulewidth}{0pt}
\renewcommand{\footrulewidth}{0pt}

% ---------------------------
% TikZ, схемы, графики
% ---------------------------
\usepackage{pgfplots}
\pgfplotsset{compat=1.18}
\usepackage[european,american inductors]{circuitikz}

% ---------------------------
% Единицы СИ
% ---------------------------
\usepackage{siunitx}
\InputIfFileExists{siunitx.cfg}{}{}

% ---------------------------
% Изображения и химия
% ---------------------------
\usepackage{subfig}
\usepackage[version=4]{mhchem}

% ---------------------------
% Цвета и комментарии
% ---------------------------
\usepackage{comment}
\usepackage[table]{xcolor}

\usepackage{bookmark}
\usepackage{microtype}
\sloppy
\emergencystretch=1em